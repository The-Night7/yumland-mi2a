\documentclass[12pt,a4paper]{article}

% --- Packages de base ---
\usepackage[utf8]{inputenc}
\usepackage[T1]{fontenc}
\usepackage[french]{babel}
\usepackage{geometry}
\geometry{margin=2cm, left=3cm} 
\usepackage{xcolor}
\usepackage{titlesec}
\usepackage{array}
\usepackage{colortbl}
\usepackage{enumitem}
\usepackage{fontawesome5}
\usepackage{graphicx}
\usepackage{tcolorbox}

% --- Couleurs Officielles ---
\definecolor{GrillRed}{HTML}{D32F2F}
\definecolor{CoalBlack}{HTML}{2D2D2D}
\definecolor{SauceCream}{HTML}{FDFBF7}
\definecolor{FryGold}{HTML}{FFC107}
\definecolor{StoneGray}{HTML}{BDBDBD}
\definecolor{DarkRed}{HTML}{B71C1C} % Pour le hover
\definecolor{BrownVisited}{HTML}{5D4037} % Pour les liens visités

% --- Configuration de la page ---
\pagecolor{white}
\setlength{\parindent}{0pt}

% Style des titres
\titleformat{\section}{\color{GrillRed}\normalfont\Large\bfseries\uppercase}{}{0em}{}[\titlerule]
\titleformat{\subsection}{\color{CoalBlack}\normalfont\large\bfseries}{}{0em}{\faCaretRight\quad}

% Bandeau latéral décoratif
\usepackage{background}
\backgroundsetup{
scale=1, angle=0, opacity=1, contents={
    \begin{tikzpicture}[remember picture, overlay]
        \path [fill=CoalBlack] (current page.north west) rectangle ([xshift=1cm]current page.south west);
        \path [fill=GrillRed] (current page.north west) rectangle ([xshift=0.2cm]current page.south west);
    \end{tikzpicture}}
}

\begin{document}

% --- HEADER ---
\begin{center}
    \includegraphics[width=5cm]{logo-le-grand-miam-removebg-preview.png} \\
    \vspace{0.4cm}
    {\Huge \textbf{\color{CoalBlack}DOCUMENT DE \color{GrillRed}CONCEPTION}} \\
    \vspace{0.2cm}
    \texttt{\small PROJET YUMLAND - PHASE 1 | LE GRAND MIAM} \\
    \vspace{0.2cm}
    \textit{\small Thème : Steakhouse \& Grillades Premium} \\
    \vspace{0.4cm}
    \textbf{Groupe MI2A} \\
    {\normalsize Myriam BENSAÏD \quad Sheryne OUARGHI-MHIRI \quad Kylian VANDEL} \\
    
\end{center}

\section{1. Identité Visuelle et Concept}

Le Grand Miam est un restaurant convivial spécialisé dans les viandes grillées et les portions généreuses. L'identité visuelle repose sur l'univers du "Grill" : chaleureux, robuste et appétissant.

\begin{tcolorbox}[colback=SauceCream, colframe=CoalBlack, arc=2mm, boxrule=1pt]
    \textbf{La Stratégie :} "Le Terroir pour Tous" \\
    \textbf{La Gestion des Viandes :}
    \begin{itemize}[label=\textcolor{GrillRed}{\faCheck}, leftmargin=*]
        \item \textbf{Bœuf :} Mise en avant de l'origine France (Charolais/Limousin). Il est indiqué clairement : \textit{"Toutes nos pièces de bœuf sont disponibles en version Halal sur demande."}
        \item \textbf{Porc :} Présent dans une section isolée "Spécialités Charcutières" pour bien séparer les produits.
    \end{itemize}
    \textbf{L'interface est conçue pour être :}
    \begin{itemize}[label=\textcolor{GrillRed}{\faCheck}, leftmargin=*]
        \item \textbf{Gourmande :} Mise en avant de visuels de haute qualité.
        \item \textbf{Accessible :} Contrastes forts pour une lisibilité parfaite (client et livreur).
        \item \textbf{Efficace :} Parcours utilisateur simplifié pour la commande rapide.
    \end{itemize}
\end{tcolorbox}

\section{2. Palette de Couleurs}
L'ambiance chromatique joue sur le contraste entre la chaleur du grill et la sobriété du charbon.

\begin{center}
\renewcommand{\arraystretch}{1.5}
\begin{tabular}{|m{3.5cm}|c|>{\centering\arraybackslash}m{2cm}|m{7cm}|}
    \hline
    \rowcolor{CoalBlack} \textcolor{white}{\textbf{Nom}} & \textcolor{white}{\textbf{HEX}} & \textcolor{white}{\textbf{Aperçu}} & \textcolor{white}{\textbf{Usage Principal}} \\
    \hline
    \textbf{Rouge "Grill"} & \#D32F2F & \cellcolor{GrillRed} & Couleur Primaire. Boutons (CTA), prix, titres, icônes. \\
    \hline
    \textbf{Noir "Charbon"} & \#2D2D2D & \cellcolor{CoalBlack} & Couleur Secondaire. Header, Footer, navigation, textes. \\
    \hline
    \textbf{Crème "Sauce"} & \#FDFBF7 & \cellcolor{SauceCream} & Arrière-plan (Body). Réduit la fatigue visuelle. \\
    \hline
    \textbf{Or "Frites"} & \#FFC107 & \cellcolor{FryGold} & Accent. Étoiles, promotions, focus formulaires. \\
    \hline
    \textbf{Gris "Pierre"} & \#BDBDBD & \cellcolor{StoneGray} & Éléments passifs. Bordures, placeholders, textes II. \\
    \hline
\end{tabular}
\end{center}

\section{3. Typographie (Google Fonts)}
Combinaison de polices libres de droits pour allier caractère et lisibilité.

\begin{tcolorbox}[colback=SauceCream, colframe=CoalBlack!20, arc=2mm, title=\color{CoalBlack}\textbf{Titres (H1, H2, H3) : OSWALD}, fonttitle=\small]
    {\Large Grand par le goût, géant par l'appétit.} \\
    \vspace{0.1cm}
    \small \textbf{Style :} Sans-serif, Condensé, Épais (Bold / 700). \\
    \textbf{Usage :} Titres de pages, noms des plats, slogans. Donne un aspect "enseigne" et "industriel" propre aux steakhouses.
\end{tcolorbox}

\begin{tcolorbox}[colback=white, colframe=CoalBlack!20, arc=2mm, title=\color{CoalBlack}\textbf{Corps de Texte : LATO} (Fallback : Roboto), fonttitle=\small]
    {\large Toutes nos viandes sont sélectionnées avec soin auprès d'éleveurs locaux.} \\
    \vspace{0.1cm}
    \small \textbf{Style :} Sans-serif, Rond, Régulier (Regular / 400). \\
    \textbf{Usage :} Paragraphes, liens, inputs, mentions légales. \\
    \textbf{Taille de base :} 16px (Desktop) / 18px (Mobile \& Livreur).
\end{tcolorbox}

\section{4. Éléments d'Interface (UI Kit)}

\subsection{Boutons (CTA)}
\begin{center}
    \begin{tcolorbox}[hbox, colback=GrillRed, colframe=DarkRed, arc=4px, boxrule=1.5pt, left=20pt, right=20pt]
        \textcolor{SauceCream}{\textbf{\faPlusCircle \quad COMMANDER}}
    \end{tcolorbox}
\end{center}
\begin{itemize}[label=\textcolor{CoalBlack}{\tiny\faCircle}, itemsep=0pt]
    \item \textbf{Forme :} Légèrement arrondis (\texttt{border-radius: 4px}). Ni trop ronds, ni trop carrés.
    \item \textbf{État Normal :} Fond \#D32F2F, Texte \#FDFBF7. Ombre légère (\texttt{box-shadow}).
    \item \textbf{État Survol (Hover) :} Fond \#B71C1C, élévation (\texttt{transform: scale(1.02)}).
\end{itemize}

\subsection{Formulaires \& Liens}
\begin{itemize}[label=\textcolor{CoalBlack}{\tiny\faCircle}, itemsep=0pt]
    \item \textbf{Champs :} Fond blanc sur fond de page crème. Bordures \#BDBDBD au repos.
    \item \textbf{Focus Saisie :} Bordure \#FFC107 (Or) et lueur externe (\texttt{box-shadow}).
    \item \textbf{Liens :} \textbf{Défaut} (Souligné/gras \#2D2D2D) | \textbf{Visité} (Brun \#5D4037) | \textbf{Survol} (\#D32F2F sans soulignement).
\end{itemize}

\section{5. Ergonomie Spécifique : Profil Livreur \faMotorcycle}
Pour répondre aux contraintes du livreur (smartphone, gants, extérieur) :
\begin{tcolorbox}[colback=white, colframe=GrillRed, arc=0mm, boxrule=1pt]
    \begin{itemize}[label=\textcolor{GrillRed}{\faHandPointer}]
        \item \textbf{Zone de Toucher (Touch Target) :} Hauteur minimale de \textbf{60px} pour tous les éléments interactifs.
        \item \textbf{Espacement :} Marges de \textbf{20px} minimum pour éviter les erreurs ("Fat finger").
        \item \textbf{Contraste :} Texte noir sur fond blanc ou bouton rouge vif pour une lisibilité maximale en plein soleil.
    \end{itemize}
\end{tcolorbox}

\section{6. Maquettes et Structure (Wireframes)}
\begin{itemize}[label=\textcolor{CoalBlack}{\faLayerGroup}]
    \item \textbf{Header :} Logo à gauche, Navigation centrée, Bouton "Connexion/Panier" à droite.
    \item \textbf{Cartes Produits :} Visuel du plat en haut, Nom et description au centre, Prix et bouton d'ajout en bas. Effet d'ombre portée (\texttt{box-shadow}) pour simuler une assiette posée sur une table.
    \item \textbf{Navigation Mobile :} Menu "Burger" simplifié pour les petits écrans.
\end{itemize}

\end{document}